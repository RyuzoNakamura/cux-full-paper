\begin{center}
  {\Large
    論 文 要 旨\\
    \vspace{2\zh}
    環境情報のセンシングを用いた他者の\\環境評価を表現するロボットシステム\\
    \vspace{2\zh}
  }
\end{center}

%   アブスト本文
本研究では、生活環境における他者の環境評価を表現するロボットシステムを提案する。我々は日常において、人間、動植物、モノなど多様な「他者」と共存している。他者はそれぞれ独自の感覚や基準で環境を評価しているが、我々がそれを直接知ることは難しい。本システムでは、環境センサーで取得したデータを他者基準で評価し、その評価結果をロボットの身体的動作を通じて表現する。具体的には、M5StickCによるセンシングとtoioロボットの動作制御を組み合わせ、人間(気温18-28℃)、猫(気温30-38℃)、バナナ(気温14-20℃、湿度45-85\%)、衣服(湿度65\%以下)など異なる他者の環境認識を表現するシステムを実装した。コンポーネント指向アーキテクチャを採用することで、システムの拡張性と保守性を確保し、「弱いロボット」のコンセプトを取り入れた動作表現により、ユーザーに自然な気づきを与えることを目指した。自己評価および展示でのフィードバックから、動作の表現力や他者の識別性などの課題も明らかになったが、本システムは生活環境における多様な他者理解を支援する新たな手法として、今後の発展可能性を示した。介護シーン、遠隔ワーク環境、植物ケアなど具体的なユースケースへの応用可能性についても検討を行った。

\vspace{3\zh}

\begin{flushright}
  宮城大学 事業構想学群 価値創造デザイン学類 感性情報デザインコース\\
  22120165\\ % 学籍番号
  中村 龍造\\ % 著者
\end{flushright}
